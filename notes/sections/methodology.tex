\section{Methodology}
\label{sec:methodology}

\BL{E}xploratory data analysis is the first step into the project, after
identifying the problem. We use visualization and descriptive statistics to
summarize the most important information to have insights. After this, we make
data preprocessing, what includes missing data imputation, outlies, and
feature engineering.   

We utilize the mean absolute error (MAE), the root
mean squared error (RMSE), and the normalized RMSE (nRMSE) as measures to
compare the models. The methods were trained using 70\% of the available data,
considering the first years. The software used for performing this experimental phase was developed
in Python (version 3.9), mainly using the Pandas and Scikit-learn.

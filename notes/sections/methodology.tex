\section{Methodology}
\label{sec:methodology}

\BL{E}xploratory data analysis is the first step into the project, after
identifying the problem. The proposed EDA includes visualization and descriptive statistics to
summarize the most relevant information to have insights. After this, we make
data preprocessing, which includes missing data imputation, outlies, and
feature engineering.   

The evaluation methods for the algorithms are the mean absolute error (MAE), the root
mean squared error (RMSE), and the rsquared (R$^2$). The methods were
trained using 70\% of the available data, considering the first years. The
software used for performing this experimental phase was developed in Python
(v. 3.9), mainly using the Pandas \cite{pandas1,pandas2}, Numpy
\cite{numpy}, Scikit-learn \cite{scikit-learn}, and Statsmodels \cite{statsmodels}. 

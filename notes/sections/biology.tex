\section{Background of air pollution}

Air pollution is a mixture of particles and gases, often not visible to human
eyes. The visible forms are widely known, such as smoke, soot and mold.
\com{The quality air index, calculated based on the levels of the gases, is
available in a diary report}, as presented in image XXX.  

\subsection{Polluting gases}

The atmosphere of the Earth is a dynamic and complex system of natural gases,
which are necessary to life \cite{gases}. The planet has a
defense mechanism that absorbs part of these fases, what forms a cycle.
However, high levels of gases concentration can cause several effects in the
living beings. The polluting gases include: 

\begin{itemize}
    \item \textbf{Óxidos de Carbono:} O monóxido de carbono (CO) é oriundo da
    combustão incompleta e não apresenta cheiro ou cor. Já o dióxido de
    carbono é um gás que contribui para o efeito estufa e, em excesso na
    atmosfera, devido à queima de combustíveis fósseis, pode causas sérios
    danos.  
    \item \textbf{Óxidos de Nitrogênio:} Também emitidos por veículos e
    tem uma aparência marrom. O dióxido de nitrogênio é um dos gases mais
    perigosos para a poluição do ar, e sua toxidade é facilmente
    identificável. 
    \item \textbf{Óxidos de Enxofre:} Causa primária da chuva ácida, muito
    comum na Europa. É natural após erupções vulcânicas. É uma forte causa de
    problemas respiratórios. 
    \item \textbf{Ozônio:} O gás ozônio ($O_3$) contém três átomos de oxigênio. Até pequenas
    concentrações desse gás são consideradas tóxicas e explosivas. Ele ocorre
    naturalmente na atmosfera, porém em pequenas quantidades, quando absorve
    radiação ultravioleta. Em condições especiais, óxidos de nitrogênio e
    hidrocarbonos podem produzir ozônio em concentração alta o suficiente para
    causar irritação nos olhos e na mucosa. 
\end{itemize}



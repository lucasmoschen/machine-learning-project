\section{Exploratory data analysis}
\label{sec:eda}

\subsection{Data description}

The dataset used in this study was extracted from the project MonitorAr
\cite{dataset-rio-ar-quality}. The table contains hourly data observations, separated by
pollutant, weather condition, and monitoring stations' characteristics from
the city of Rio de Janeiro. Table
\ref{tab:measured-data} informs the most important variables used, and Table
\ref{tab:pollutants-measured} indicates the measured pollutants. The events were collected between January 1,
2011,
and March 31, 2021. A total of 661,662 records were used. 

\begin{table*}[t]
    \centering
    \begin{tabular}{c|c|c|c|}
        \cline{2-4}
                                                                                 & \textbf{Name} & \textbf{Type} & \textbf{Description}              \\ \hline
        \multicolumn{1}{|c|}{\multirow{7}{*}{\textbf{Meterological conditions}}} & Chuva         & float         & Rainfall (mm)                     \\ \cline{2-4} 
        \multicolumn{1}{|c|}{}                                                   & Pres          & float         & Atmospheric Pressure (mbar)       \\ \cline{2-4} 
        \multicolumn{1}{|c|}{}                                                   & RS            & float         & Solar radiation (w/m2)            \\ \cline{2-4} 
        \multicolumn{1}{|c|}{}                                                   & Temp          & float         & Temperature (°C)                  \\ \cline{2-4} 
        \multicolumn{1}{|c|}{}                                                   & UR            & float         & Relative humidity (\%)            \\ \cline{2-4} 
        \multicolumn{1}{|c|}{}                                                   & Dir\_Vento    & float         & Wind direction (°)                \\ \cline{2-4} 
        \multicolumn{1}{|c|}{}                                                   & Vel\_Vento    & float         & Wind speed (m/s)                  \\ \hline
        \multicolumn{1}{|c|}{\multirow{5}{*}{\textbf{Measurement conditions}}}   & Data          & datetime      & Measurement date and hour         \\ \cline{2-4} 
        \multicolumn{1}{|c|}{}                                                   & CodNum        & ineger        & Number of the monitoring station  \\ \cline{2-4} 
        \multicolumn{1}{|c|}{}                                                   & Estação       & string        & Name of the monitoring station    \\ \cline{2-4} 
        \multicolumn{1}{|c|}{}                                                   & Lat           & float         & Latitude position of the station  \\ \cline{2-4} 
        \multicolumn{1}{|c|}{}                                                   & Lon           & float         & Longitude position of the station \\ \hline
        \end{tabular}
    \caption{Measured parameters by the program MonitorAr.}
    \label{tab:measured-data}
\end{table*}

\begin{table*}[b]
    \centering
    \begin{tabular}{|c|c|}
    \hline
    \textbf{Monitoring station}        & \textbf{Measured gases/particulates}              \\ \hline
    Centro (CA)             & O$_3$, CO, PM$_{10}$                              \\ \hline
    Copacabana (AV)         & SO$_2$, O$_3$, CO, PM$_{10}$                      \\ \hline
    São Cristóvão (SC)      & SO$_2$, O$_3$, CO, PM$_{10}$                      \\ \hline
    Tijuca (SP)             & SO$_2$, NOx, O$_3$, CO, PM$_{10}$                 \\ \hline
    Irajá (IR)              & SO$_2$, NOx, O$_3$, CO, HC, PM$_{2.5}$, PM$_{10}$ \\ \hline
    Bangu (BG)              & SO$_2$, NOx, O$_3$, CO, HC, PM$_{10}$             \\ \hline
    Campo Grande (CG)       & SO$_2$, NOx, O$_3$, CO, HC, PM$_{10}$             \\ \hline
    Pedra de Guaratiba (PG) & O$_3$, PM$_{10}$                                  \\ \hline
    \end{tabular}
    \caption{Pollutant data measured by each monitoring station. CO and HC are measured in (ppm), while the others are measured in (µg/m3).}
    \label{tab:pollutants-measured}
\end{table*}

\begin{enumerate}
    \item Reportar valores nulos da chuva e índice de missing values
\end{enumerate}

\begin{enumerate}
    \item Gráficos dos gases e interpretação de alguns deles. Analisar curtose
    e assimetria.
    \item Mensurações temporais de alguns gases. Selecionar alguns poucos
    \item Testes de estacionariedade nas séries utilizadas. 
    \item Mais alguns gráficos de visualização. 
\end{enumerate}

\subsection{Data preprocessing}

The data preprocessing is an important step before the usage of machine
learning algorithms, in order to report robust and neat results. 

\begin{enumerate}
    \item Imputation of missing data
    \item Handling outliers
    \item normalization and standardization. 
    \item feature engineering
\end{enumerate}

\subsubsection{Missing data imputation}

In this dataset, there is two types of missing data: (1) monitoring stations
do not measure all pollutants by construction. For instance, it is not
measured NOx in Centro and Copacabana; and (2) monitoring stations did not
measure in a period for some reason. We have to deal with them in two
different ways. 

\begin{enumerate}
    \item Possíveis formas de imputação: estimação polinomial de 2ª ordem.
    Alguns testes simples pode ser interessante. Para locais onde não há
    estimação, não faz sentido imputar. 
\end{enumerate}

\subsubsection{Data transformation}

\begin{enumerate}
    \item Transformação Yeo-Johnson 
\end{enumerate}

\subsubsection{Feature extraction}

From the variable {\tt Data}, we can observe (Figure
\ref{fig:histogram-obs-years}) that 2011 has less observations, because there
were only four of the eight stations operating. For that reason, we do not
consider the data from this year. 

\begin{figure}
    \begin{center}
        \includegraphics[width=0.47\textwidth]{histogram.eps}
    \end{center}
    \caption{Number of hourly measurements per year. In 2011, only half of the monitoring stations worked.}
    \label{fig:histogram-obs-years}
\end{figure}

\begin{enumerate}
    \item Analisar sazonalidade. Adicionar termo seno e cosseno de forma que
    exista sazonalidade diária, isto é, 
    $$
    \text{hour\_sin} = \sin(2\pi \text{ hour}/24)
    $$
    \item Create variable season. 
    \item Visualize the series autocorrelation in order to define the number
    of lag variables per pollutant and particle.
    \item Complete the dataset with temporal variables.     
    \item The number of features is XXX: lag features, roll mean features,
    weekend, season, trigonometric  
\end{enumerate}

\subsubsection{Feature selection}

In order to reduce the dimensionality of the feature space, from the XXX
variables described, variable selection was performed. Filters are methods
which perform feature selection, as well as embedding methods, although they
depend on the machine learning algorithm chosen. 

\begin{enumerate}
    \item Pearson correlation-based filter to feature selection. Observe
    correlation of the lag variables. 
\end{enumerate}

The objective of this work is generate different forecasting models for the
pollutants A, B, C, and D. The maintained features in the dataset can be seen
in Table XXX.

\com{We applied PCA in order to reduce the dimensionality.}


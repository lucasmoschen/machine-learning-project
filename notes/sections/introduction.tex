\section{Introduction}
\label{sec:introduction}

\BL{A}ir quality is a growing concern and area
of research, because most of the cities world-wide have been facing problems with it in
the past few decades \cite{mayer1999}. The rapid increase of the urban
population and the development of the cities is causing environmental
pollution, what can give rise to damage on human health. 

The emission and
transmission of air pollutants, such as, Nitrogen dioxide (NO$_2$), Carbon
monoxide (CO), Ozone (O$_3$)
and Particulate Matter (PM), results in the ambient air pollution, which can be
caused by different factors. The World Health
Organization (WHO) explained \cite{who2006} that PM, O$_3$, and NO$_2$ have, respectively, the
strongest effects on health of air quality. 

In Rio de Janeiro city, the city hall recognized the problem and created the
Program MonitorAr-Rio in 2008 \cite{relatorio2011}. The objective was to
monitor the air quality in the city and inform the results to the population.
Eight fixed stations monitor the main pollutants defined in the legislation,
and some meterological conditions. 

It is important to have updated knowledge and accurate predictions of the
air pollutants, in order to help the formulation of public health and
environmental policies. This study proposes models for hourly/daily air
quality forecasting for the city of Rio de Janeiro, using the algorithms XXX
and XXX. 

The text is organized as follows. Section \ref{sec:problem-definition} defines
the problem clearly and mathematically. Section \ref{sec:background} gives a
background on the topic of air quality and air pollution. Section \ref{sec:methodology} presents
the methodology of the work. We present an exploratory data analysis with the
data path in Section \ref{sec:eda}. Section \ref{sec:experiments} presents the
methods used and the experiments related to each one. Sections
\ref{sec:results}, \ref{sec:discussion}, and \ref{sec:conclusion} presents the
results and concludes it. 


\section{Problem definition}
\label{sec:problem-definition}

\BL We want to produce time predictions of some pollutants for the Rio de Janeiro
city, considering the weather, location, and time variables. We also want to
develop a method for estimating the air quality of not monitored regions by
the program based on the monitored ones.

Let $Y_i$ be the random
variable indicating the quantity of i$^{th}$ pollutant measured in a specific
monitoring station, for instance, the quantity of ozone. If $\bb{X} \in \mathcal{X}
\subseteq \R^{p}$ is the random vector of meterological conditions measured,
$t$ is the time of measurement, and $s_1, ..., s_8$ the positions of the
monitoring stations, we have that 
\begin{equation}
    \label{eq:problem definition}
    Y_i = f_i(t, s_k, \bb{X}) + \epsilon_{t,s_k,i}
\end{equation}
where $\epsilon_{t,s,i}$ is a random variable with mean zero. After
observing, for $t_1 < ... < t_n$,  
$$y_i^{t_1,s}, ..., y_i^{t_n,s}$$ 
and 
$$
\bb{x}^{t_1,s}, ..., \bb{x}^{t_n,s},
$$
we want to make predictions $y_i^{\bar{t},s}$ for $\bar{t} > t_{n+1}$. We also want to
predict $y_i^{t, \tilde{s}}$ for $\tilde{s}$ different of $s_k$, for $k =
1,...,8$. 
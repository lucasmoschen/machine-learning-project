\section{Results}
\label{sec:results}

The results are separated by gas, but not by monitoring station. We present
more detailed results for only one (Tijuca station was chosen randomly), and after aggregated results
considering all the stations. This is done because we have more than
a hundred models to analyse. To deal with this diversity, the steps
(hyperparameters choice and evaluation) are automated. We are
making predictions one hour ahead and it is possible to compare with one day
ahead models. 

\subsection{Tijuca monitoring station}

The results follow the order specified in the above summary. 

\subsubsection{CO}

\subsubsection{O\texorpdfstring{$_3$}{3}}

\subsubsection{PM\texorpdfstring{$_{10}$}{10}}

\subsubsection{AIQ}

\subsection{Aggregated results}

\subsubsection{CO}

\subsubsection{O\texorpdfstring{$_3$}{3}}

\subsubsection{PM\texorpdfstring{$_{10}$}{10}}

\subsubsection{AIQ}

\subsection{Model for other locations}

Here, we want to make predictions about pollutant levels at other not measured
sites. Given that each location has a specific model, the prediction
is the weighted mean regarding each prediction 



\begin{enumerate}
    \item 1 modelo para cada gás e cada estação. 
    \item Temos 7 opções de modelos até o momento (analisando friamente, 23 x
    7 = 161 modelos a serem fittados e analisados). 
    \item Como realizar tantos experimentos para cada modelo de forma
    automática? \com{A ideia é escrever um código para escolha de hiperparâmetros e
    reporte de resultados automaticamente.}
    \item Pegar uma estação para analisar os resultados mais cuidadosamente. 
    \item Testes de significância.  
\end{enumerate}

Lembrar de 

\begin{enumerate}
    \item Testar estacionaridade de cada série; 
    \item Lembrar de inverter os dados pelo power transformation: ler p2, fazer power transform. 
\end{enumerate}



\section{Discussion and Future work}
\label{sec:discussion}


\section{Conclusion}
\label{sec:conclusion}

Poluição do ar é uma mistura de partículas e gases, muitas vezes
não visíveis aos olhos humanos. As formas visíveis são amplamente 
conhecidas, como fumaça, fuligem e mofo. Um índice utilizado para 
medir os níves de poluição do ar é o índice de Qualidade do Ar (IQA), 
que é disponibilizado pelo boletim diário, como visto na imagem \ref{boletim}.

\subsection{Gases Poluentes}

A atmosfera da Terra consiste em um sistema dinâmico e complexo de gases
naturais que são necessários para a vida, de acordo com \cite{gases}. O
planeta apresenta um mecanismo de defesa que absorve parte desses gases,
fazendo com que um ciclo seja formado. Entretanto, altos níveis de
concentração de gases podem causar diversos efeitos nos organismos vivos,
variando de gás para gás. 

Dentre as principais fontes dos poluentes, estão a queima de combustíveis
fósseis, queima de mata e áreas de agricultura. Os gases poluentes incluem: 

\begin{itemize}
    \item \textbf{Óxidos de Carbono:} O monóxido de carbono (CO) é oriundo da
    combustão incompleta e não apresenta cheiro ou cor. Já o dióxido de
    carbono é um gás que contribui para o efeito estufa e, em excesso na
    atmosfera, devido à queima de combustíveis fósseis, pode causas sérios
    danos.  
    \item \textbf{Óxidos de Nitrogênio:} Também emitidos por veículos e
    tem uma aparência marrom. O dióxido de nitrogênio é um dos gases mais
    perigososos para a poluição do ar, e sua toxidade é facilmente
    identificável. 
    \item \textbf{Óxidos de Enxofre:} Causa primária da chuva ácida, muito
    comum na Europa. É natural após erupções vulcânicas. É uma forte causa de
    problemas respiratórios. 
\end{itemize}

Na próxima subsessão, trataremos do gás ozônio, o ator principal da nossa
análise, segundo \cite{ozone}. 

\subsection{O gás Ozônio}

O gás ozônio ($O_3$) contém três átomos de oxigênio. Até pequenas
concentrações desse gás são consideradas tóxicas e explosivas. Ele ocorre
naturalmente na atmosfera, porém em pequenas quantidades, quando absorve
radiação ultravioleta. Em condições especiais, óxidos de nitrogênio e
hidrocarbonos podem produzir ozônio em concontração alta o suficiente para
causar irritação nos olhos e na mucosa. 
\documentclass[fleqn,10pt]{SelfArx} % Document font size and equations flushed left
\usepackage{lettrine}
\usepackage[portuguese]{babel} % Specify a different language here - english by default
\usepackage{graphicx}

\setlength{\columnsep}{0.55cm} % Distance between the two columns of text
\setlength{\fboxrule}{0.75pt} % Width of the border around the abstract
\usepackage{lipsum}
\definecolor{color1}{RGB}{0,0,90} % Color of the article title and sections
\definecolor{color2}{RGB}{0,20,20} % Color of the boxes behind the abstract and headings

\PassOptionsToPackage{hyphens}{url}\usepackage{hyperref} % Required for hyperlinks
\hypersetup{hidelinks,colorlinks,breaklinks=true,urlcolor=color2,citecolor=color1,linkcolor=color1,bookmarksopen=false,pdftitle={Title},pdfauthor={Author}}

\setlength{\headheight}{40pt}

\JournalInfo{Modelagem de Fenômenos Biológicos, 2019} % Journal information
\Archive{Avaliação 2} % Additional notes (e.g. copyright, DOI, review/research article)

\PaperTitle{Concentração de Ozônio 
na região de Bangu, Rio de Janeiro} % Article title
\PaperSubTitle{Uma abordagem em séries temporais}

\Authors{Lucas Machado Moschen\textsuperscript{1}*} % Authors
\affiliation{\textsuperscript{1}\textit{Fundação Getulio Vargas - Escola de Matemática Aplicada}} % Author affiliation
\affiliation{*\textbf{Corresponding author}: lucas.machadomoschen@gmail.com} % Corresponding author

\Keywords{Poluição do ar --- Ozônio --- Predição} % Keywords - if you don't want any simply remove all the text between the curly brackets
\newcommand{\keywordname}{Palavras-chave} % Defines the keywords heading name

\Abstract{A poluição do ar é um dos temas mais imporantes da contemporaneidade
e, este trabalho, pretende recortar esse problema na região de Bangu, no Rio
de Janeiro. A abordagem foi consttruída com o ferramental de séries temporais,
utilizando conceitos como regressão linear, autocorrelação, estacionariedade e
tempo, obviamente. Os resultados não foram tão bons quanto os esperados, mas
oo processo para alcance foi bem estruturado, com testes de hipóteses e
Cross-Validation. A maior parte do teoria do trabalho foi contruída com o
auxílio do livro Forecasting, devidamente refrenciado. Gráficos foram gerados
para elucidar a questão e trazer esse problema tão próximo à nossam mesa de
discussão. }

%----------------------------------------------------------------------------------------

\begin{document}

\flushbottom % Makes all text pages the same height

\maketitle % Print the title and abstract box

\tableofcontents % Print the contents section

\thispagestyle{empty} % Removes page numbering from the first page

\section*{Introdução} % The \section*{} command stops section numbering

\addcontentsline{toc}{section}{Introdução} % Adds this section to the table of contents

\section{Introduction}
\label{sec:introduction}

\BL{A}ir quality is a growing concern and area
of research, because most of the cities world-wide have been facing problems with it in
the past few decades \cite{mayer1999}. The rapid increase of the urban
population and the development of the cities is causing environmental
pollution, what can give rise to damage on human health. 

The emission and
transmission of air pollutants, such as, Nitrogen dioxide (NO$_2$), Carbon
monoxide (CO), Ozone (O$_3$)
and Particulate Matter (PM), results in the ambient air pollution, which can be
caused by different factors. The World Health
Organization (WHO) explained \cite{who2006} that PM, O$_3$, and NO$_2$ have, respectively, the
strongest effects on health of air quality. 

In Rio de Janeiro city, the city hall recognized the problem and created the
Program MonitorAr-Rio in 2008 \cite{relatorio2011}. The objective was to
monitor the air quality in the city and inform the results to the population.
Eight fixed stations monitor the main pollutants defined in the legislation,
and some meterological conditions. 

It is important to have updated knowledge and accurate predictions of the
air pollutants, in order to help the formulation of public health and
environmental policies. This study proposes models for hourly/daily air
quality forecasting for the city of Rio de Janeiro, using the algorithms XXX
and XXX. 

The text is organized as follows. Section \ref{sec:problem-definition} defines
the problem clearly and mathematically. Section \ref{sec:background} gives a
background on the topic of air quality and air pollution. Section \ref{sec:methodology} presents
the methodology of the work. We present an exploratory data analysis with the
data path in Section \ref{sec:eda}. Section \ref{sec:experiments} presents the
methods used and the experiments related to each one. Sections
\ref{sec:results}, \ref{sec:discussion}, and \ref{sec:conclusion} presents the
results and concludes it. 


\section{Problem definition}
\label{sec:problem-definition}

\BL We want to produce time predictions of some pollutants [See Section
\ref{sec:polluting-gases} for a detailed description] for the Rio de Janeiro
city, considering the weather, location, and time variables. We also want to
develop a method for estimating the air quality of not monitored regions by
the program based on the monitored ones.

Let $Y_i$ be the random
variable indicating the quantity of i$^{th}$ pollutant measured in a specific
monitoring station, for instance, the quantity of ozone. If $\bb{X} \in \mathcal{X}
\subseteq \R^{p}$ is the random vector of meterological conditions measured,
$t$ is the time of measurement, and $s_1, ..., s_8$ the positions of the
monitoring stations, we have that 
\begin{equation}
    \label{eq:problem definition}
    Y_i = f_i(t, s_k, \bb{X}) + \epsilon_{t,s_k,i}
\end{equation}
where $\epsilon_{t,s,i}$ is a random variable with mean zero. After
observing, for $t_1 < ... < t_n$,  
$$y_i^{t_1,s}, ..., y_i^{t_n,s}$$ 
and 
$$
\bb{x}^{t_1,s}, ..., \bb{x}^{t_n,s},
$$
we want to predict $y_i^{\bar{t},s}$ for $\bar{t} > t_{n+1}$. We also want to
predict $y_i^{t, \tilde{s}}$ for $\tilde{s}$ different from $s_k$, for $k =
1,...,8$. 

\section{Poluição do Ar}

\section{Background of air pollution}
\label{sec:background}

\BL{A}ir pollution is a mixture of particles and gases, often not visible to human
eyes. The visible forms are widely known, such as smoke, soot and mold.
According to Conselho Nacional do Meio Ambiente (CONAMA)
\cite{conama-air-pollution}, atmospheric pollution is 

\begin{quotation}
    ``... any form of matter or energy with intensity and in quantity,
    concentration, time or characteristics in disagreement with the
    established levels, and that make or may make the air: inappropriate,
    inconvenient, harmful to the environment or harmful to safety.''
\end{quotation}

Different anthropogenic process emit air pollutants, such as, for instance,
fossil fuels (motor traffic and domestic), agriculture, and industries.
Besides that, natural disaster are an important gas emitter, not controllable,
though. 

\subsection{Polluting gases}
\label{sec:polluting-gases}

The atmosphere of the Earth is a dynamic and complex system of natural gases,
which are necessary to life \cite{gases}. The planet has a defense mechanism
that absorbs part of them. However, high levels of gases concentration can cause several effects in the
living beings. Air pollutants are divided according to their origin \cite{who2006}: 

\begin{itemize}
    \item {\bf Primary:} those emitted into the atmosphere from a source; or
    \item {\bf Secondary:} those formed within the atmosphere through a
    chemical reaction.
\end{itemize}

Other important distinctions are related to their chemical class - organic or
inorganic, and related to their physical state - gaseous or particulate. The
selected pollutants are: 

\begin{itemize}
    \item \textbf{Carbon monoxide (CO):} CO results of incomplete combustion
    of matter with carbon and it does not present smell or color. Its
    concentration level is strongly related to car traffic, in addition to
    agricultural and forest fires. When breathed, it reduces the ability of
    oxygen transport by the hemoglobins.  

    \item \textbf{Ozônio (O$_3$):} It exists naturally in the atmosphere,
    where it has the function of absorbing ultraviolet radiation from the sun
    and of reducing its impact on the planet Earth surface. Nitrogen oxides and
    organic compounds, with oxygen and high temperatures form the ozone. However, in
    the troposphere, it is toxic and even explosive. Its
    effect on health can be drastic. 

    \item \textbf{Particulate matter (PM$_{10}$):} It is composed by particles
    of solid or liquid matter suspended in the air, with 10 micrometers of
    less. Combustion is one of the major human sources of particulate matter.
    Its effects include respiratory tract infections and damages to the
    environment. When the aerodynamic diameter is between 2.5 and 10
    micrometers, we call it PM$_10$, and it is the most common among the
    suspended particles measured.  

\end{itemize}

Other measured pollutants that we will not study by computacional and data
limitations are Sulfur Dioxide (SO$_2$), Nitrogen Oxides (NOx), and Hydrocarbons (HC).

\subsection{Urban air quality problems}

Air pollution episodes, such as, for example, the 1930 Meuse Valley in Belgium
\cite{nemery2001}, the 1952 Great Fog of London
\cite{polivka2018}, and the 2006 Southeast Asian Haze \cite{jones2006} raised
questions and concerns about high levels of pollutants in urban ambients,
which caused several deaths and hospitalizations. The rapidly expanding
populations and cities also increase the exposure to them.

\com{Deaths caused in Brazil}

\subsection{Air quality index}

The air quality index is a synthetic indicator what simplifies the divulgation
and the communication among population, private and public sectors, ONGs, etc.
Its divulgation is made through diary reports for each monitoring
station\footnote{\url{http://jeap.rio.rj.gov.br/je-metinfosmac/boletim}} as
shown in Figure \ref{fig:boletim}. 

\com{How is it calculated? Present the five levels}

\begin{figure*}[!ht]  
    \includegraphics[width=\linewidth]{../images/boletim01-03-2018.png}
    \caption{Report from March 1$^{st}$, 2018.}
    \label{fig:boletim}
\end{figure*}





\section{Análise da Série Temporal}

Uma série temporal é uma coleção de observações $x_t$ registrada no tempo $t$.
Esse tempo pode ser discreto ou contínuo. Na nossa análise, o tempo é contínuo,
entretanto, interpretamos ele como discreto, já que as medições são guardadas a
cada hora. O objetivo de analisar a série temporal é tentar compactar a
informação disponibilizada pela prefeitura para interpretação a posteriori,
estudar a relação com outras variáveis medidas e prever futuros valores usando
algum modelo. Nesse caso, mostraremos mais de um. Observe a série temporal de
Ozônio no ano de 2018 na Figura \ref{time-series}. Grande parte desse texto
também é contido no livro Econometria de Séries Temporais \cite{econometria}. 

\begin{figure}[!t]
    \includegraphics[width=\linewidth]{img/graphic4.png}
    \caption{Concentração de Ozônio em 2018.}
    \label{time-series}
\end{figure}

\subsection{Autocorrelação}

Precisamos encontrar padrões na série temporal, a fim de encontrar possíveis
tendências, não necessariamente lineares, sazonalidades ou mudanças cíclicas. Para
isso, utilizo uma medida de relação linear entre valores com atraso da série.
Definimos $r_k$ como essa medida entre os valores $y_t$ e $y_{t - k}$, para
todos os valores de $y_t$ capturados. Esse método é derivado da correlação de
Pearson. 

\begin{equation}
    \label{autocorrelation}
    r_k = \frac{\sum_{t = k + 1}^T (y_t - \overline{y})(y_{t - k} -
    \overline{y})}{\sum_{k=1}^T (y_t - \overline{y})^2}    
\end{equation}

A partir disso, é possível gerar a função de autocorrelação. Neste trabalho,
explorei essa função em relação aos meses e às horas. Em relação aos meses, meu
$k$ varia entre os valores de $0$ a $12$. De fato $r_0 = 1$. Espera-se que a
autocorrelação seja zero se o tempo não tenha influência sobre os dados. Nesse
caso, chamamos a série de ruído branco. Claro que como o conjunto de dados tem
tamanho finito, a autocorrelação dificilmente será $0$. Desta maneira, para
esses casos, é esperado que os valores estejam entre os limites $\pm
\frac{2}{\sqrt{T}}$ com probabilidade $95\%$, onde $T$ é o tamanho da amostra.
A figura \ref{acf-months} representa a função. Note o quão insignificantes se
tornam os limites. 

Valores altos para pequenos atrasos indicam tendência na série, já que
existe uma correlação alta entre os valores de um mês com os valores do mês
anterior. Entretanto, como o decréscimo não é suave, a tendência não é tão
observada. Também é interessante observar o efeito da sazonalidade em períodos
de 6 meses. Isso pode estar relacionado a período mais quente e frio, e a
alternância, nesses casos, da concentração de ozônio. 

No caso das horas, também foi observada alta correlação para valores de
atraso pequenos, mas o resultado foi pouco elucidativo. 

\begin{figure}  
    \includegraphics[width=\linewidth]{img/graphic3.png}
    \caption{Função de autocorrelação em relação aos meses. As linhas azuis
    representam os limites. }
    \label{acf-months}
\end{figure}

\subsection{Estacionariedade}

Uma série é dita estacionária quando ao passar do tempo, seus valores
mantem-se ao redor da média e variância constante. Quando uma série apresenta
tendência, ela não é estacionária. A importância de uma série estacionária é
para a realização do modelo, já que vários deles são descritos sobre séries
estacionárias. Para testar se nossa série temporal é estacionária,
consideremos o teste Dickey-Fuller Aumentado, um tipo de teste de uma raíz,
uma causa para a não estacionariedade.

$$H_0 : ~there~is~unit~root~(non~stationary)$$
$$H_1 : ~there~is~no~unit~root~(stationary)$$

Considero o nível de significância de 5\%. O p-valor do teste esteve na ordem
de $10^{-30}$, e a hipótese nula foi rejeitada. O teste foi realizado em
Python, através da função \textit{adfuller} do módulo para análide de séries
temporais \textit{statsmodels.tsa}. Para esse teste em Python, conferir
\cite{adf-test}. Outro fator importante de se analisar é a ergodicidade,
porém, nesse ensaio, assumirei a série como engórdica. 

\section{Modelando Séries Temporais}

Para modelar uma série temporal, a fim de fazer futuras predições, existem
diversos métodos apresentados pela literatura. Os dados consideram as máxima
média a cada 8h, que são consideradas para o cálculo do IQA no banco de dados.
Para os valores faltantes, preenchi com o valor anterior e deixo para futuros
trabalhos estudar métodos mais eficazes. Os dados não são afetados de forma
significativa por inflação, mudança na população ou calendário, pois o
intervalo temporal em anos não é muito grande. 

Para realizar os seguintes métodos, desenvolvo um processo para a análise do
resultado com o diagnóstico dos dos resíduos através do teste para
autocorrelação de \textit{Portmanteau}. Nesse teste, testamos se as primeiras $h$
autocorrelações são significantemente diferentes da esperada em ruído branco.
Existem várias estatísticas para esse teste, como, por exemplo, o teste
\textit{Box-Pierce} e, mais preciso, o teste \textit{Ljung-Box}, conforme
apêndice A. 

Para avaliar a precisão de nossas previsões, utilizo o método de
Cross-Validation para dividir os dados, mais de uma vez, em treino e teste. Ao
fim, para cada divisão temporal, calculo a precisão do modelo através do
método da Média Absoluta de Erros de Escala (Veja apêndice B) e, depois, faço
a média desse cálculo de previsão. 

\textbf{Métodos Estudados nesse trabalho: }
\begin{enumerate}
    \item Método da Média;
    \item Regressão Linear;
    \item Suavização Exponencial; 
\end{enumerate}

\subsection{Método da Média}

Nesse método de previsão, o modelo simplesmente afirma que a próxima
observação é a média das observações anteriores. Ele tem sua importância para
teste de sanidade e verificação dos métodos de análise de resultado.

\subsection{Regressão Linear}

A regressão linear admite que exista uma relação linear entre duas ou mais
séries temporais. Assim, 
$$y_t = \beta_0 + \beta_1x_{1,t} + ... + \beta_nx_{n,t}+ \epsilon_t$$
A variável $\epsilon$ captura tudo que as variáveis escolhidas não capturam.
Observe que os parâmetros indicam o quanto as variáveis são relacionadas e
como elas se relacionam (positivamente ou negativamente). Assumimos algumas
coisas sobre os erros: 

\begin{itemize}
    \item A média $0$;
    \item São não autocorrelacionados. Caso não fossem, existiria informação
    adicional que não foi extraída dos dados;
    \item São não relacionados com as variáveis preditoras, que representam as
    variáveis $x_{i,t}$. 
    \item É interessante, mas não necessário, que os erros sejam normalmente
    distribuídos. 
\end{itemize}

O método que utilizo para escolher os parâmetros é a estimação de mínimos
quadrados. Também, para esse modelo, lanço mão do coefieciente de
determinação, o $R^2$ (Veja Apêncice C).

As variáveis que utilizarei para esse modelo são Temperatura, Radiação Solar, 
Velocidade do Vento e Umidade Relativa. Essas quatro variáveis apresentam os
maiores valores absolutos de correlação e tem relação direta com a formação do
ozônio. A radiação solar, é um exemplo já construído na literatura e tem
relação direta com o ozônio \cite{solarRadiation}. 

Além disso, eu crio 11 variáveis indicadoras para os meses, a fim de capturar
a sazonalidade ou tendências em certos períodos. É importante dizer que não é 
necessária mais uma variável, pois ela estará incluída no parâmetro de
interceptação, que não acompanha variáveis independentes. Ao colocar essa
variável, podemos criar uma \textit{variável indicadora armadilha}. Essas
variáveis indicam $1$ para o mês correspondente e $0$ caso contrário. 

\subsection{Suavização Exponencial}

Proposta no final dos anos de 1950, por Brown, Holt e Winters, a suavização
exponenicial tem motivado diversos sucessos em previsões. Basicamente, esse
método faz uma média ponderada das observações passadas, só que esses pesos
decaem exponencialmente com o tempo. Especificamente, os pesos decrescem com
uma razão geométrica. Nesse método, existem diversas variações. Chamamos de
Suavização Exponencial Simples a seguinte equação:

$$\hat{y}_{T+1|T} = \sum_{j=0}^{T-1} \alpha(1 - \alpha)^jy_{T-j} + (1-
\alpha)l_0,$$
onde, os parâmetros $\alpha$ e $l_0$ devem ser estimados. $l_0$ a primeira
estimação, enquanto $\alpha$ é um valor entre $0$ e $1$ que indica o quando de
importância se dá aos eventos passados. 

Entretanto, nesse trabalho, desenvolvo um modelo para capturar a sazonalidade
dos dados, onde existe a equação de previsão e três equações de suavização: o
componente de tendência, o componente sazonal e o nível da suavização. Nesse
caso, existem mais dois parâmetros a serem estimados, além de um parâmetro de
sazonalidade que indica o número de períodos do ano. Esse modelo foi
desenvolvido por Holt e Winters. 

A fim de encontrar os parâmetros, também utilizo o método de minimização de
quadrados dos erros. 

\section{Resultados e Discussões}

\section{Results}

\begin{enumerate}
    \item Quantitative results of my experiments. 
    \item Statistical significance.
\end{enumerate}

\section{Discussion}


\section{Future work}


\section{Conclusion}


\phantomsection

\section{Apêndice}

\input{apendices.tex}

\section*{Agradecimentos} 

Agradeço ao professor Cláudio pela indicação do Livro em questão 
e agradeço à prefeitura do Rio por organizar dados tão importantes
sobre a cidade e disponibilizá-los de maneira simples para o 
entendimento. 

\addcontentsline{toc}{section}{Agradecimentos}


\phantomsection
\bibliographystyle{unsrt}
\bibliography{sample}

\end{document}